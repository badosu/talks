\documentclass{beamer}
\usetheme{Copenhagen}
\usepackage[utf8]{inputenc}
\usepackage[brazil]{babel}  % Pacote
\usepackage{mflogo}
\usepackage{caption}
\usepackage{minted}
\author{Amadeus Folego}
\title{Ferramentas Científicas em Ruby}
\date{}
\institute{
  \begin{tabular}{c c}
    {\em email} & \url{amadeusfolego[at]gmail[dot]com}\\
    {\em github} & \url{@badosu}\\
    {\em twitter} & \url{@badosu\_}
  \end{tabular}
}
\begin{document}
  \frame[plain]{\titlepage}
  \begin{frame}{Porquê?}
    \begin{itemize}
      \item Visualizar e extrair informações associadas a um grande conjunto de dados numéricos \pause
      \item Manipulação de imagens, efetuar transformadas \pause
      \item Simular modelos, otimizar parâmetros, minimizar funções \pause
      \item Inferir estatísticas
    \end{itemize}
  \end{frame}
  \begin{frame}{Características Desejadas}
    \begin{itemize}
      \item Manipulação de arrays numéricos de alta performance \pause
      \item REPL (Read Print Eval Loop) prático \pause
      \item Ferramentas de visualização (plotting)
    \end{itemize}
  \end{frame}
  \begin{frame}{Caso: Python/SciPy}
    \begin{itemize}
      \item Ferramenta de visualização: matplotlib \pause
      \item Manipulação numérica: numpy \pause
      \item REPL: ipython -pylab
    \end{itemize}
  \end{frame}
  \begin{frame}[fragile]{NArray}
    \begin{itemize}
      \item Prover estruturas numéricas \pause
      \item https://github.com/masa16/narray (v0.6) \pause
      \item Manipula valores em estruturas numéricas nativas do C \pause
      \item Dados: int (8,16,32), float (32,64) e complexo (64,128) \pause
      \item Entre 28-50 vezes mais rápido \pause
      \item Consome 4 vezes menos memória, em média, do que Array \pause
      \item Pouco documentado, faltam features básicas \pause
      \item Implementado em extensões em C, não compatível com JRuby
    \end{itemize}
\end{frame}
  \begin{frame}[fragile]{NArray vs Array}
    \begin{minted}{ruby}
# teste de performance
n = 10**8
    \end{minted}
    \pause
    \begin{minted}{ruby}
# Array
a = (1..n).to_a
b = (1..n).to_a
(1..n).map{|i| a[i]*b[i]}
    \end{minted}
    \pause
    \begin{minted}{ruby}
# NArray
NArray.int(n).indgen * NArray.int(n).indgen
    \end{minted}
    \pause
    \begin{itemize}
      \item O segundo exemplo performou em 2.3s \pause
      \item O primeiro começou a travar a máquina após 4m30s
    \end{itemize}
\end{frame}
  \begin{frame}{SciRuby}
    \begin{itemize}
      \item Ambição de prover capacidade científica similar ao SciPy \pause
      \item Deve utilizar projetos como NMatrix em cima do ATLAS \pause
      \item Última atualização há 6 meses \pause
      \item Projeto muito ambicioso, perdeu tração, mas com bons frutos: \pause
      \begin{itemize}
        \item Visualization: rubyvis \pause
        \item Statistics: statsample \pause
        \item Numeric: minimization, integration, nmatrix \pause
        \item Probability: distribution
      \end{itemize}
    \end{itemize}
  \end{frame}
  \begin{frame}{NMatrix}
    \begin{itemize}
      \item Ambição de prover capacidade científica similar ao SciPy \pause
    \end{itemize}
  \end{frame}
  \begin{frame}[fragile]{Borel}
    \begin{itemize}
      \item Operações em conjuntos ordenados \pause
      \item Único requisito é incluir Comparable, e compatibilidade numérica para alguns métodos \pause
    \end{itemize}
    \pause
    \begin{minted}{ruby}
Interval[1,5] | Interval[7,9] == Interval[[1,5],[7,9]]
    \end{minted}
    \pause
    \begin{minted}{ruby}
Interval['a','f'] ^ Interval['c','g'] == Interval['c','f']
    \end{minted}
\end{frame}
  \begin{frame}[plain]
    \titlepage
  \end{frame}
\end{document}
