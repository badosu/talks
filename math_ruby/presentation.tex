\documentclass{beamer}
\usetheme{Copenhagen}
\usepackage[utf8]{inputenc}
\usepackage[brazil]{babel}  % Pacote
\usepackage{mflogo}
\usepackage{caption}
\usepackage{minted}
\author{Amadeus Folego}
\title{Ferramentas Científicas em Ruby}
\date{}
\institute{
  \begin{tabular}{c c}
    {\em email} & \url{amadeusfolego[at]gmail[dot]com}\\
    {\em github} & \url{@badosu}\\
    {\em twitter} & \url{@badosu\_}
  \end{tabular}
}
\begin{document}
  \frame[plain]{ \titlepage }
  \begin{frame}{Caso: Python/SciPy}
    \begin{itemize}
      \item matplotlib \pause
      \item numpy \pause
      \item ipython -pylab
    \end{itemize}
  \end{frame}
  \begin{frame}[fragile]{NArray}
    \begin{itemize}
      \item Prover estruturas numéricas \pause
      \item https://github.com/masa16/narray (v0.6) \pause
      \item Manipula valores em estruturas numéricas nativas do C \pause
      \item Dados: int (8,16,32), float (32,64) e complexo (64,128) \pause
      \item Entre 28-50 vezes mais rápido \pause
      \item Consome 4 vezes menos memória, em média, do que Array \pause
      \item Pouco documentado, faltam features básicas \pause
      \item Implementado em extensões em C, não compatível com JRuby
    \end{itemize}
\end{frame}
  \begin{frame}[fragile]{NArray vs Array}
    \begin{minted}{ruby}
# teste de performance
n = 10**8
    \end{minted}
    \pause
    \begin{minted}{ruby}
# Array
a = (1..n).to\_a
b = (1..n).to\_a
(1..n).map{|i| a[i]*b[i]}
    \end{minted}
    \pause
    \begin{minted}{ruby}
# NArray
NArray.int(n).indgen * NArray.int(n).indgen
    \end{minted}
    \pause
    \begin{itemize}
      \item O segundo exemplo performou em 2.3s \pause
      \item O primeiro começou a travar a máquina após 4m30s
    \end{itemize}
\end{frame}
\begin{frame}{SciRuby}
  \begin{itemize}
    \item Ambição de prover capacidade científica similar ao SciPy \pause
    \item Deve utilizar projetos como NArray e NMatrix em cima do ATLAS \pause
    \item Última atualização há 6 meses
  \end{itemize}
\end{frame}
  \begin{frame}[plain]
    \titlepage
  \end{frame}
\end{document}
